\documentclass{acmsiggraph}               % final

%% These two line bring in essential packages: ``mathptmx'' for Type 1
%% typefaces, and ``graphicx'' for inclusion of EPS figures.

\usepackage{graphicx}
\usepackage{url}
\usepackage{times}
\usepackage{ifpdf}

%% Redundant.
%% Paper title.

%% \title{C++ News Server and Client Implementation}

%% \author{Gustaf Waldemarson\thanks{e-mail: ada09gwa@student.lu.se}
%% \and Martin Trasteby\thanks{e-mail: atf09dtr@student.lu.se}
%% \and Erik Jansson\thanks{ada09eja@student.lu.se}
%% \and Tommy Olofsson \thanks{ada09tol@student.lu.se}}
%% \affiliation{Faculty of Engineering (LTH), Lund University \\ Sweden}


%% Keywords that describe your work.
\keywords{C++, C, News Server, Database, Remote Connections}


\newcommand{\HRule}{\rule{\linewidth}{0.5mm}}

%%%%%% START OF THE PAPER %%%%%%
\begin{document}

\ifpdf
  \DeclareGraphicsExtensions{.jpg,.pdf,.mps,.png}
\else
  \DeclareGraphicsExtensions{.eps}
\fi

\begin{titlepage}

\begin{center}

% Upper part of the page %
\textsc{\LARGE University of Lund}\\[1.5cm]

% Includes an SVG image made in inkscape, requires inkscape to be installed 
% and included in the enviroment path! Didn't get centered for some reason...
%\begin{center}
%  \begin{figure}
%    \centering
%    \def\svgwidth{\columnwidth}
%    \includesvg{StarComposition}
%  \end{figure}
%\end{center}

\textsc{\Large Image Analysis - FMA170}\\[0.5cm]
% Title
\HRule \\[0.4cm]
{ \huge \bfseries Handin 4}\\[0.4cm]

\HRule \\[1.5cm]
% Author
\begin{minipage}{0.4\textwidth}
\begin{flushleft} \large
\emph{Author:}\\
Gustaf \textsc{Waldemarson}\\
\emph{SAM-ID:}\\
ada09gwa\\
\emph{Social Security Number:}\\
900428-0930
\end{flushleft}
\end{minipage}

\vfill

{\large \today}

\end{center}

\end{titlepage}


\maketitle

\begin{abstract}
This report details an implementation of a news server with a fixed
communication protocol, as well as a client using this protocol in
order to retrieve or create news on the server. The articles and
groupings of articles called "news groups" are stored in two different
versions of a database -- one stores the information on the primary
memory, while another stores them on the hard drive. The client is a
terminal application.

\end{abstract}
\keywordlist

\section{Server, memory only database}
\label{sec:smemdb}
%% TODO: Subsections for classes used.

\section{Server, disk backed database}
See \ref{sec:smemdb}. Another database class are employed to save the
database to disk \ref{ref:class_memdb}.
\label{sec:sdiskdb}
%% TODO: Subsections for memdb.

\section{Client}
\label{sec:client}
%% TODO: Subsections for classes used. Ref. to already described classes.

\end{document}



